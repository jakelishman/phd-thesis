\chapter{Certification of Higher-Order Coherence}\label{sec:coherence}

\begin{coauthorship}
The experimental work in this chapter was carried out by the ion-trapping group at Imperial College London.
I calculated all of the measurement statistics to derive the unbiased estimators, and found the optimal state-creation and projection sequences numerically.
Florian Mintert and I proved the robustness of the interference-pattern certifier under general measurement operators, and I performed all the numerical optimisations to verify the new threshold values.
The work in this chapter was also described in \intextcite{Corfield2021}.
\end{coauthorship}

The manipulation and distillation of entanglement was recognised early as essential to quantum computation~\cite{Bennett1996}, leading to efforts to move beyond Bell-inequality tests to more discriminating methods of detecting entanglement~\cite{Peres1996,Horodecki1996} that have only multiplied with time~\cite{Horodecki2009,Horodecki2021}.
Coherence, however, has only more recently been recognised as a resource in the same manner~\cite{Levi2014,Baumgratz2014,Streltsov2017}, with its uses now being known for applications varying from quantum information processing~\cite{Hillery2016,Shi2017}, to the creation of nonequilibrium entropy~\cite{Santos2019} and the extraction of thermodynamic work~\cite{Korzekwa2016}.

A hierarchy for quantum coherence was defined in \cref{sec:qi-basic-definitions}, where a pure state $\ket\psi$ is said to be coherent in a particular basis $\{\ket j\}$ if its representation $\ket\psi = \sum_j c_j \ket j$ has at least two nonzero coefficients $c_j$.
This readily extends to multiple levels, where a pure state is called $k$-coherent, or said to have a coherence rank of $k$, if it has at least $k$ non-zero coefficients.
Mixed states are $k$-coherent if all possible pure-state decompositions include at least one $k$-coherent element.
Despite having this simple classification scheme, it is not trivial to continuously quantify coherence in the general case.
It feels logical that $\bigl(\ket0 + \ket1\bigr)/\sqrt2$ should be somehow more coherent than $\sqrt{0.1}\ket0 + \sqrt{0.9}\ket1$ due to the greater imbalance between the two basis states, yet this idea is harder to justify at higher levels of coherence, or with density operators that may have multiple possible decompositions.
Direct measurement is typically fraught, as the only measurement basis available is usually the basis over which the coherence is defined.
Still, it is valuable to be able to classify the degree of coherence of a system; higher-order coherence is its own resource, which may be expended to enhance phase-discrimination tasks~\cite{Ringbauer2018,Castellini2019}.

Assuming perfectly coherent operations and lossless measurements in the coherence basis, it would be possible to determine the coherence of a state by complete reconstruction of its density operator.
Even with these perfect conditions, extensive time is required to build up sufficiently precise statistics on the elements, and for high orders of coherence in mixed states the subsequent classical analysis is similarly difficult.
These obstacles motivate a different approach, just as they do in entanglement verification.
Instead of inferring the state, we can instead take a different measurement whose sole purpose is to distinguish states with different orders of coherence.

These coherence certifiers are entirely analogous to entanglement witnesses, although there are significant further complications in the former.
Entanglement may be detected by coherent local operations and classical communications, but its generation requires operations outside this set~\cite{Peres1996}.
Coherence, however, is detected and generated by the same set of operations.
This appears to enforce an unfortunate circular requirement that any measurement to verify the preparation of a coherent superposition must trust the same operations it is assessing.

This chapter describes a robust high-order coherence certifier that overcomes the issues both in scalability of measurements and in impossible assumptions of the na\"ive approach.
It works when even the basis of coherence is not accessible to measurement, and is demonstrated by an experimental realisation in the motional state of a trapped ion in collaboration with the group at Imperial.
The metric is provably immune to false positives, and requires only simple functions of a one-dimensional interference pattern.
It is built on prior work out of the Imperial quantum information theory group~\cite{Dive2020}, with significant extra effort required to make the scheme valid for general quantum measurements that cannot even distinguish the coherence basis states.


\section{Quantum coherence}

The concept of coherence itself is not unique to quantum mechanics.
Classical coherence is a core component of the wave theory of electromagnetism.
Quantum coherence, on the other hand, allows for single particles to interfere with themselves.
Along with entanglement and quantisation itself, it plays a central role in quantum effects, from the complete description of the laser to the celebrated Hong--Ou--Mandel dip~\cite{Hong1987}.

The quantification of this coherence was first explored by noting that two-level coherence may be characterised by the magnitude of the off-diagonal term in the density operator, and then generalising this across pairs of orthogonal subspaces that spanned the entire Hilbert space~\cite{Aberg2006}.
While this initial effort was not considered as such, the popularity of resource theories~\cite{Chitambar2019} led to more complete works to fit coherence into this new framework~\cite{Levi2014,Baumgratz2014}.
The principal additions are the identification of a set of incoherent operations that cannot increase coherence, analogous to the local operations and classical communication with entanglement, and a requirement that a measure is convex under the mixing of states.
Both are physically motivated: one cannot conjure coherence from nothing, nor can they increase it by classically mixing two states together.
Formally, the incoherent operations are the quantum channels that map the set of incoherent states to itself, and a coherence measure $\operatorname C$ is convex if
\begin{equation}\label{eq:coherence-convexity}
\lambda \operatorname C[\op\rho_1] + (1 - \lambda) \operatorname C[\op\rho_2] \ge \operatorname C[\lambda\op\rho_1 + (1-\lambda)\op\rho_2],
\end{equation}
for a mixing parameter $0\le\lambda\le1$ and all pairs of density operators $\op\rho_1$ and $\op\rho_2$.

The convexity condition is reminiscent of the triangle inequality, and indeed distance-based measures are perhaps the most studied of coherence quantifiers from a theoretical perspective~\cite{Streltsov2017}.
The insight is to quantify the coherence of a state by the minimum distance between it and an incoherent state.
This permits a family of measures, based on the particulars of the distance used.
Various works have investigated the properties of using the relative entropy~\cite{Baumgratz2014,Winter2016}, the state infidelity~\cite{Streltsov2015} and both Schatten-$p$ and $\ell_p$ matrix norms~\cite{Baumgratz2014,Rana2016}.

Measures and witnesses of higher-order coherence are more complex to construct, as they target an understanding of a more nuanced view of coherence.
As an illustration, the $\ell_1$ norm is commonly used to characterise two-level coherence, but plainly cannot distinguish the number of superposition elements taken alone.
More involved functionals of the off-diagonal terms have been used to construct more complete quantifiers~\cite{Levi2014,Ringbauer2018}, but these still require detailed tomographic measurements.
Some of these schemes may be used to construct witnesses, or to use incomplete data to lower-bound the rank of coherence present~\cite{Ringbauer2018}, but these prior methods make significant assumptions about the properties of the physical system and the correctness of coherent manipulations.
Instead, we turn to interference-pattern methods, which are well-known as a standard indicator of coherence in two-level systems.


\section{Interference-pattern methods}

Ramsey-type experiments have long been a standard tool to verify coherence between two states.
In these, the system is first subjected to some coherent operation that ought to create a superposition between the basis states.
It evolves freely by a varied amount, then the reverse of the original coherent operation is applied and the population of one of the two states is measured.
Regardless of how faithfully the coherent operation was performed, evidence of oscillation in the resulting data is proof that coherence must have existed; an incoherent state would be unable to interfere with itself during the reverse mapping.
The simplicity of these experiments and the minimal amount of data required make them very attractive as a base for inference of further system properties~\cite{Oi2006}.
We will consider how they may be used to classify multilevel coherence.

In a two-level Ramsey experiment, the two states are allowed to evolve relative to each other for a complete period.
For photonic systems, this is equivalent to creating a path difference between the two basis states and applying a controllable phase shift to only one of the paths before recombination.
Physically this describes a Mach--Zehnder interferometer.
In higher dimensions, one can imagine a generalisation of such an interferometer as having a different path and phase shift for each basis state~\cite{vonPrillwitz2015}.
This creates an interference pattern over as many variables as there are dimensions in the Hilbert space.
Notably, while any evidence of periodic oscillation in a two-level Ramsey experiment proves some degree of coherence, one cannot distinguish true higher-order coherence from an incoherent mixture of pairwise coherent systems by simply counting the frequency components.
The maximal peak-to-peak visibility does, however, encode some information about the rank of the coherence of the state; there are threshold values for each rank of coherence that no lower-ranked coherent state can exceed.

In arbitrarily many dimensions, optimising to find the global maximum becomes experimentally taxing.
It is similar in principle to general state tomography methods.
To ease the computational burden, one can instead consider only lower-order moments $\{M_n\}$ of an interference pattern, defined by
\begin{equation}\label{eq:coherence-kai-moment}
M_n(\op\rho) = \int\!\mathrm dw(\vec\phi)\,{\matel{\vec\phi}{\op\rho}{\vec\phi}}^n,
\quad\text{with}\quad \ket{\vec\phi} = \frac1{\sqrt d}\sum_j e^{-i\phi_j}\ket{\phi_j},
\end{equation}
and some measure $w(\vec\phi)$ that acts as a prior.
The uniform prior ${(2\pi)}^{-d}$ is appropriate for completely unknown input states of dimension $d$, but others can be used to reduce the sampling requirements in the discretisation of the integral when approximating it experimentally.
Lower-order moments vary less rapidly with respect to the state, and so require fewer measurements to approximate well.
Calculating $\lim_{n\to\infty} M_n^{1/n}$ is equivalent to finding the maximal peak-to-peak visibility, which has the theoretically best distinguishing characteristics.
The lower moments still present the same thresholded structure, but with lower distinguishability~\cite{vonPrillwitz2015}.
The moments also satisfy the convexity property of \cref{eq:coherence-convexity}, a critical condition for them to certify coherence in the context of a resource theory.
However, retrieving any of this information requires either that projective measurements of arbitrary states can be taken accurately, or that individual basis states can be phase-shifted independently with completely reliable coherent manipulations.
These severely limit the viability of this multi-dimensional approach.

These problems were addressed by prior work out of the Imperial theory group~\cite{Dive2020}.
This considers any space whose coherence basis vectors $\{\ket j\}$ can be made to evolve under the equally spaced Hamiltonian $\H \propto \sum_j j\proj jj$.
If necessary, one can also expand the Hilbert space with intermediate dummy states that are never populated to achieve the equal separation.
This restriction on the required evolution replaces the projection onto an arbitrary state $\ket{\vec\phi}$ in \cref{eq:coherence-kai-moment} with a free evolution $\U_{\text f}(\phi) = \sum_j e^{-ij\phi}\proj jj$ followed by some fixed mapping sequence $\U_{\text m}$, making the interference pattern one-dimensional.
The Hamiltonian naturally occurs as the free evolution in harmonic-oscillator systems, but can be effectively driven in many others, including those with degeneracy.
For example, in a system of $d$ qubits and the coherence defined over the product of $z$-basis eigenstates, the required evolution can be realised by implementing
\begin{equation}
\U(\phi) = \R_1(\phi)\R_2(2\phi) \dotsb \R_d\bigl(2^{d-1}\phi\bigr)\quad\text{where}\quad\R_k(\phi) = \exp\Bigl(i\phi\sz^{(k)}\Bigr).
\end{equation}
The operators $\R$ are rotations around the individual qubits' Pauli $Z$ axes, and the whole evolution requires only single-qubit operations.
This form of evolution is frequently implemented virtually, in an error-free manner.
Control pulses generally evolve synchronously with the basis-state phase evolution, in which case applying a constant phase shift to the driving fields is equivalent to separate evolution.

In this simpler system, the interference pattern for projective measurements onto $\ket\chi$ becomes
\begin{equation}\label{eq:coherence-interference-pattern-projector}
p(\phi) = \matel[\big]{\chi}{\mkern2mu\U_{\text m}\U_{\text f}(\phi)\mkern1mu\op\rho\mkern3mu\U_{\smash f}^\dagger\mkern-1mu(\phi)\U_{\text m}^\dagger}{\chi}.
\end{equation}
With only a single dimension, it is now always feasible to evaluate the entire interference pattern, so the prior from \cref{eq:coherence-kai-moment} can be replaced with the standard uniform distribution, leaving the moments as
\begin{equation}\label{eq:coherence-moment}
M_n = \frac1{2\pi}\int_0^{2\pi}\!\mathrm d\phi\,p(\phi)^n.
\end{equation}
These moments alone no longer provide the desired threshold structure; they are all maximised to unity by taking $\op\rho = \proj\chi\chi$, which is completely incoherent.
Instead, \citet{Dive2020} showed that the family of ratios of moments
\begin{equation}\label{eq:coherence-rn}
R_n = \frac{M_n}{M_1^{n-1}}
\end{equation}
satisfy all the necessary conditions to be certifiers of higher-order coherence: they are convex in both arguments by the definition of \cref{eq:coherence-convexity}, and they have hierarchical threshold values such that the maximal value of $R_n$ with a $k$-coherent state is strictly less than the maximal value with a $\msquash{(k+1)}$-coherent state.
All ratios with $n>2$ are capable of certifying all ranks of coherence.
In practice, we will only use $R_3$ for the same experimental reason that lower moments are generally preferred: they can be well approximated with fewer measurements.

Proving the convexity of the $\{R_n\}$ certifiers over the tested state is largely straightforward.
The pattern $p(\op\rho,\,\phi)$ is always between zero and one for all normalised states, and $p^n$ is trivially convex for non-negative $p$.
The moments $M_n$ are therefore convex as integration with respect to $\phi$ is linear with respect to the state.
Showing the convexity of $R_n$ can then be achieved by showing that the second derivative
\begin{equation}
\partial_\lambda^2 R_n\bigl(\lambda \op\rho_1 + (1-\lambda)\op\rho_2\bigr) \ge 0 \quad\text{for $0\le\lambda\le1$},
\end{equation}
since the function is asymptote free.
This can be evaluated in terms of the moments---dropping the function arguments for clarity---to give
\begin{align}
\partial_\lambda^2R_n &= \frac1{M_1^{n+1}}\Bigl[M_1^2\bigl(\partial_\lambda^2M_n\bigr) - 2\msquash{(n-1)}M_1\bigl(\partial_\lambda M_1\bigr)\bigl(\partial_\lambda M_n\bigr) + n\msquash{(n-1)}M_n{\bigl(\partial_\lambda M_1\bigr)}^2\Bigr],\\
\intertext{which can be reduced to a trivially positive form}
&= \frac{n(n-1)}{M_1^{n+1}}\Bigl\langle p^{n-2}{\bigl[p\langle\partial_\lambda p\rangle - \langle p\rangle\bigl(\partial_\lambda p\bigr)\bigr]}^2\Bigr\rangle
\end{align}
using the notation $\langle f\rangle = \int_0^{2\pi}\mathrm d\phi\,f(\phi)/(2\pi)$, the explicit derivatives
\begin{equation}
\partial_\lambda M_n = n\bigl\langle p^{n-1}\bigl(\partial_\lambda p\bigr)\bigr\rangle
\quad\text{and}\quad
\partial_\lambda^2 M_n = n(n-1)\Bigl\langle p^{n-2}\bigl(\partial_\lambda p\bigr)^2\Bigr\rangle,
\end{equation}
and rearrangements of the form $\langle c\rangle = c$ for interference-phase-independent $c$.

It is generally intractable to analytically calculate the hierarchical threshold values implied by the convexity of the certifier and the convexity of the set of $k$-coherent states.
The maximum value of any $R_n$ for an incoherent state is trivially one, and \citet{Dive2020} showed that the maximal value of $R_3$ for 2-coherence is $\text{\sfrac{5}{\kern-.03em 4}} = 1.25$ and upper-bounded the maximal value for 3-coherent states to $\text{\sfrac{179}{96}} \approx1.86$.
They also performed thorough numerical optimisations---I personally replicated these using the techniques described later---to find empirical upper bounds on the actual attainable maxima for each coherence rank: 1.25, 1.77, 2.32 and 2.88 were the largest observed values for 2-, 3-, 4- and 5-coherent states respectively.
It is worth highlighting that $R_3$ is not a measure of coherence, but a form of witness; it is not necessarily zero for an incoherent state, and there are incoherent states that exhibit larger values of $R_3$ than coherent states.
Our intent is to use it to certify balanced coherent superpositions as being unambiguously of the desired rank, rather than to produce a perfectly discriminating measure.

The convexity of the certifier with respect to the input state and the convexity of the sets of $k$-coherent states make $R_3$ a valid witness.
However, this is not sufficient to make the certifier valid in cases where the coherent mapping operation $\U_{\text m}$ cannot be trusted.
This is an unfortunate problem for most certifiers, since the mapping is invariably implemented with the same operations that prepare the state to be tested, so if the state cannot be prepared coherently---what this certifier purports to test---then the likelihood of $\U_{\text m}$ functioning correctly is also high.
It is therefore imperative that a faulty mapping to the measurement basis cannot increase the coherence.
\citet{Dive2020} showed this for the case that a projective measurement onto the state $\U_{\text m}\ket\chi$ was replaced by a probabilistic projection onto one of a set of states $\ket{\chi_j}$ with probability $p_j$, with the proof progressing near-identically to the proof of convexity over the tested state.
This means that under these relatively relaxed assumptions about the measurement, the $R_n$ certifiers can \emph{never} produce false positives when detecting coherence.

These assumptions do not hold in our experimental realisation of choice, though.
We seek to create high-rank coherent states in the motion of a single trapped ion.
The only available measurement is a projective measurement on the \emph{electronic} state of the ion, which is an operator of the form $\proj ee \otimes\I_{\text{motion}} = \sum_n \proj{g,n}{g,n}$.
This cannot be described as a probabilistic projection onto one of a set of states.
Instead we must turn to a more general formalism of quantum measurement to test the validity and robustness of the $R_3$ certifier in this system.


\section{General measurements}
\label{sec:coherence-general-measurements}

The most general case of operators we will consider are the positive operator-value measures (\textsc{povm}s) introduced in \cref{sec:qi-measurement}.
The interference pattern of \cref{eq:coherence-interference-pattern-projector} is generalised to
\begin{equation}\label{eq:coherence-interference-pattern-povm}
p(\phi) = \Tr\bigl[\op A\mkern1mu\U_{\text m}\U_{\text f}(\phi)\mkern1mu\op\rho\mkern3mu\U_{\smash{\text f}}^\dagger\mkern-1mu(\phi)\U_{\text m}^\dagger\bigr].
\end{equation}
Without loss of generality, we may replace $\op A$ with a unitary transformation $\U_{\text m}\op A\U_{\text m}^\dagger$ and use the cyclic property of the trace to drop the $\U_{\text m}$ terms.
Keeping them separate gives more physical intuition of the process, but is an unnecessary complication for robustness analysis.
The convexity of the certifier with respect to the input state is easily shown by the same method that was used previously, so we can immediately progress to calculating the threshold values for different ranks of coherence.

\subsection{Analytic threshold for 3-coherence}

As before, we consider a coherence basis of states $\{\ket n\}$ that can be made to evolve by $\U_{\text f}(\phi) = \sum_n e^{-in\phi}\proj nn$.
The state and the operator are decomposed in terms of these states as
\begin{equation}
\op\rho = \sum_{n,m} \rho_{nm}\proj nm \quad\text{and}\quad \op A = \sum_{n,m} A_{nm}\proj nm,
\end{equation}
where the coefficients are complex.
This form allows the interference pattern \cref{eq:coherence-interference-pattern-povm} to be rewritten as
\begin{equation}\label{eq:coherence-interference-pattern-explicit}
p(\phi) = \sum_n\rho_{nn}A_{nn} + 2\sum_{n>m}\abs{\rho_{mn}A_{nm}}\cos\bigl((n-m)\phi + \theta_{nm}\bigr),
\end{equation}
in terms of some angles $\{\theta_{nm}\}$ that are the complex phases of the $\rho_{mn}A_{nm}$ terms.
All oscillating cosine terms average to zero over the course of one period.
This makes the lowest moment, $M_1$ independent of the relative phases.
Similarly, powers of the interference pattern can all be rearranged into sums of terms of the form $\alpha \cos(\beta\phi + \theta_{n_1,m_1} \pm \theta_{n_2,m_2}\pm\dotsb)$, with $\alpha>0$ and integer $\beta$.
The only contributing terms have $\beta = 0$, consequently all moments $M_n$ and the certifier $R_3$ have a maximum when all the $\{\theta_{nm}\}$ are zero, \textit{i.e.}\ when $\op\rho$ and $\op A$ are real-symmetric matrices.
These are not the only cases when the maximum is reached, but we can proceed under this assumption without loss of generality.

We can analytically calculate the maximal value that $R_3$ can achieve for any 2-coherent state.
With the convexity trivially proven for the general measurement, we need only consider pure states.
The energy separation of the two populated coherence-basis states does not contribute to the value of $R_3$, and we will label them $\ket0$ and $\ket1$.
Since the maximum is achieved with a positive real-symmetric density operator, we may parametrise the input state as $\sqrt x\ket0 + \sqrt{1-x}\ket1$ for $0\le x\le1$.
Using \cref{eq:coherence-interference-pattern-explicit}, the explicit form of the certifier is
\begin{equation}
R_3 = xA_{00} + (1-x)A_{11} + \frac{6x(1-x)A_{01}^2}{xA_{00} + (1-x)A_{11}}.
\end{equation}
In order for the measurement operator to be a valid value in a \textsc{povm}, the two on-diagonal elements must have a maximal value of one, and the off-diagonal element must satisfy
\begin{equation}
A_{01} \le \min\bigl\{A_{00}A_{11},\,(1-A_{00})(1-A_{11})\bigr\}.
\end{equation}
The symmetry in this constraint represents the choice between using $\op A$ or $1 - \op A$ as the measurement operator, so we may examine only the branch with $A_{00} + A_{11} \le 1$.
As expected, $R_3$ is maximised when the coherence between the two basis states is maximised when the inequality in the constraint is tight, giving $A_{01} = A_{00}A_{11}$.
Clearly if any of $x$, $A_{00}$ or $A_{11}$ are zero, the system described is simply incoherent, and $R_3$ may attain a maximal value of unity.

The true maximum for 2-coherent states can be found using the method of Lagrange multipliers with the constraints $0 < \{x,\,A_{00},\,A_{11}\} < 1$ and $A_{00} + A_{11} \le 1$.
As only one bound can be tight we need only one slack variable $\lambda$, and find
\begin{equation}
\mathcal L = R_3 - \lambda(A_{00} + A_{11} - 1) \quad\text{for}\quad \lambda\ge0.
\end{equation}
The derivative with respect to $A_{00}$ is
\begin{equation}
\frac{\partial\mathcal L}{\partial A_{00}} =
    x\frac{x^2 A_{00}^2 + 2x(1-x)A_{00}A_{11} + 7{(1-x)}^2A_{11}^2}{{\bigl[xA_{00} + (1-x)A_{11}\bigr]}^2} - \lambda,
\end{equation}
which is transformed into the derivative with respect to $A_{11}$ by the transformations $x\to1-x$ and $A_{00}\leftrightarrow A_{11}$.
The fraction is strictly positive, so stationary points require that $\lambda$ is as well, in turn forcing the $A_{11} = 1 - A_{00}$ to satisfy the complementary slackness condition.
With all of these conditions, the optimal measurement operator in the restricted $\{\ket0,\,\ket1\}$ subspace can be written in matrix form as
\begin{equation}
\op A = \begin{pmatrix}
    A_{00} & \sqrt{A_{00}(1-A_{00})} \\
    \sqrt{A_{00}(1-A_{00})} & 1 - A_{00}
\end{pmatrix},
\end{equation}
which is precisely the form of a rank-1 projective measurement of the state $\sqrt{A_{00}}\ket0 + \sqrt{1 - A_{00}}\ket1$.
The problem has now been reduced to what was already shown by \citet{Dive2020}, and the maximal value of $R_3$ for 2-coherent states remains \sfrac54.
Any state that has a measured value of $R_3$ above this value must be at least 3-coherent.


\subsection{Numeric evaluation of thresholds}
\label{sec:coherence-numeric-thresholds}

Expanding this direct analysis beyond 2-coherence proves tricky.
Instead, we use numerical techniques to maximise the certifier over each space of $k$-coherent states and general measurement operators, in order to empirically find the upper bounds.
At a high level, we simply wish to take any general-purpose maximisation algorithm, such as the quasi-Newton Broyden--Fletcher--Goldfarb--Shanno (\textsc{bfgs}) method~\cite{Press2007}, and have it adjust a random \textsc{povm} value and $k$-coherent density operator to maximise $R_3$.
In practice, this means finding a parametrisation of $\ell$ elements that takes a vector in $\mathbb R^\ell$ smoothly to the search space.
Simply taking the individual matrix elements is not suitable, as the constraints on the values become highly non-linear and unsuitable for numerical optimisation.
A better way is to craft a parametrisation that is a surjection of $\mathbb R^\ell$ onto the search space; it is permissible---though somewhat undesired---for multiple vectors to correspond to the same pair of \textsc{povm} value and density operator, but it is required that \emph{all} such pairs have at least one associated parameter vector.
If this is achieved, one can use an unconstrained optimisation routine, which are typically orders of magnitude faster and more complete than those for problems with non-linear constraints.

Any \textsc{povm} value $\op A$ can be written as a sum of simple projective measurements as $\op A = \sum_j a_j \proj{\psi_j}{\psi_j}$, for some scalar constants $0\le \{a_j\} \le1$ and a set of orthonormal states $\{\ket{\psi_j}\}$.
For convenience, we will optimise separately over different numbers of non-zero $a_j$.
We approach the parametrisation problem top-down.
Each non-zero $a_j$ requires a single parameter, and any standard mapping of $\mathbb R\to[0,1]$ is suitable for the transformation: logistic transforms, arctangent transforms, and so on.
The $\{\ket{\psi_j}\}$ can be chosen by parametrising an orthonormal basis of the complete space, and selecting the desired number from the basis.
This can be done by first taking an arbitrary basis of the full Hilbert space, and parametrising a single pure state out of it.
The input basis is then limited to cover only the subspace orthogonal to the chosen state, reducing its dimensionality by one.
These steps are then repeated, each time removing a dimension from the available subspace for parametrisation by considering only the space orthogonal to all previously selected states.
Once enough states have been found, the parametrisation is complete.
The reduction of the Hilbert space to the orthogonal subspace can be done via the Gram--Schmidt process, which is suitably deterministic, stable and smooth.
A valid pure state in an $n$-dimensional Hilbert space can be generated by taking $n-1$ amplitudes $c_q$ and phases $\theta_q$, and returning the normalised dot product of the vector $(1,\,c_1e^{i\theta_1},\,c_2 e^{i\theta_2},\,\dotsc)$ with the basis.
This shows that there is some duplication in the parametrisation, but in practice it does not pose an issue.

The convexity properties of $R_3$ should make it unnecessary to draw arbitrary density matrices for the input states, in favour of using pure states, but for completeness' sake we can define a parametrisation.
All density matrices that are at most $k$-coherent can be written as a sum $\sum_j p_j\op\rho_j$, where the $\{p_j\}$ are probabilities and the individual $\op\rho_j$ are arbitrary density matrices each in their own $k$-dimensional subspace spanned by distinct choices of $k$ basis states from the coherence basis.
As with many discrete components in smooth optimisations, we handle the choice of different subspaces by simply repeating the optimisations many times for each possible set of choices.
Density operators are positive semi-definite, and thus have a Cholesky decomposition $\op\rho = \op L\op L^\dagger$ for a lower-triangular matrix $\op L$.
We can therefore parametrise an $n$-dimensional density matrix by drawing $n(n+1)/2$ parameters to be the magnitudes of the triangular matrix elements and $n(n-1)/2$ parameters to be the phases of the off-diagonal elements.
Matrices parametrised in this way will not give unity-trace density operators, and so if $\op L'$ is the parametrised triangular matrix, the output density operator is $\op\rho = \op L'\op L'^\dagger/\Tr\bigl(\op L'\op L'^\dagger\bigr)$.

A parametrisation drawn in this manner is clearly biased.
Unlike in random sampling where it is strongly preferable to draw from the Haar measure to avoid sampling artefacts, this is not a particular problem for optimisation.
The only consideration is to ensure that the optimisation landscape does not become too flat for convergence.
This is most likely to be an issue if the optimal value is achieved when certain input parameters must become close to infinite to represent the desired value in the output space, and the target cost varies slowly with respect to large changes in the inputs.
In this case, this issue does not frequently arise since the landscape is well featured and the convergence criteria can reliably be reached.

To locate the threshold values with general measurement operators, several thousand optimisations were run in parallel~\cite{ImperialHPC} in Hilbert spaces of varying dimensions, taking varying ranks of the measurement operator $\op A$ and density operators with various numbers of $k$-coherent components on different subspaces.
In all cases, the quasi-Newton method would reduce the total density matrix to a single pure state, setting all but one component probabilities $p_j$ to zero, and create a measurement operator $\op A$ that was precisely a projective measurement onto the input state.
This is exactly consistent with the expected results, and the threshold values of $R_3$ were in total agreement with the prior work~\cite{Dive2020}.
Further, fixing $\op A$ to be a higher-rank projector by requiring multiple $a_j = 1$ always resulted in a maximal value of $R_3$ that was lower than before.
Specifically, each additional rank of projector reduced the maximal value achievable by $R_3$ to the next highest threshold value, for example a rank-2 projector in a 4-dimensional Hilbert space could never measure more than 3-coherence, no matter if the input state was greater.
This is intuitive; adding extra orthogonal components reduces the distinguishability of states, which is a key component of the certification.

We have now shown that $R_3$ is a robust coherence certifier, even for the most general class of quantum measurements.
No matter how imperfectly the mapping $\U_{\text m}$ is implemented, one can never measure a value of $R_3$ above certain thresholds if the input state does not exhibit sufficient-rank coherence.
The only assumption is that any error in the mapping sequence is independent of the free-evolution phase $\phi$ being applied.
In practice, this is easily satisfied in all systems of interest.


\section{State-creation sequences}
\label{sec:coherence-creation}

The structure of our coherence-creation experiment is a generalisation of the standard two-state Ramsey experiment.
In the usual form, a single ion is ideally prepared in the $\ket{g,0}$ state, although in practice the motion typically has some small thermal component with mean phonon occupation $\bar n \ll 1$.
A superposition state $\ket{g,0} + \ket{g,1}$\footnote{%
    Throughout this chapter we will drop the normalisation factors from state descriptions for legibility.
    All states considered are actually of unit norm; there is no use of unnormalised states.
} is prepared by first applying a $\pi/2$ pulse on the carrier---creating the state $\ket{g,0} + \ket{e,0}$---followed by a $\pi$ pulse on the first red sideband.
This is allowed to evolve its phase for some time, then subject to the inverse of the state-creation sequence and measurement.
If all operations are implemented perfectly, this shows a sinusoidal response of the ground-state population to the phase evolution.
These oscillations are maximum amplitude, because the inverse of the creation sequence maps the target state back to $\ket{g,0}$ and the only orthogonal state that becomes populated during the evolution, $\ket{g,0}-\ket{g,1}$, to $\ket{e,0}$.
There are two major problems preventing the obvious generalisation to \emph{create \emph{any} superposition, evolve it, invert the creation, and measure}: arbitrary state creation is non-trivial, and inversion of the creation sequence will prove to produce an unsuitable interference pattern.

We will first deal with the creation of arbitrary motional superpositions.
The two-level Ramsey scheme works by exploiting the non-interaction of the first red sideband with the $\ket{g,0}$ state.
Directly extending this method to create superpositions with higher motional states would require access to the second-order sideband transitions and beyond.
As described in \cref{sec:iontrap-interaction}, the ion--laser coupling in typical trap configurations is insufficient to drive these interactions in a reasonable time frame without causing significant off-resonant effects on other transitions.
We must use a method that is limited to first-order sidebands, here a minor extension to previous work~\cite{Gardiner1997,Ben-Kish2003} that uses both red and blue sidebands rather than just the red.

\begin{figure*}%
    \includegraphics{coherence-superposition.pdf}%
    \caption[Creation of an arbitrary motional superposition]{\label{fig:coherence-superposition-preparation}%
        The algorithm used to produce arbitrary motional superpositions~\cite{Gardiner1997}, illustrated creating $\ket{g,0} + \ket{g,1} + \ket{g,2}$.
        Grey circles represent state occupation with size proportional to population, and the green, red and blue arrows respectively represent the carrier, red- and blue-sideband transitions.
        (a) Start from the target state.
        (b) Apply the red-sideband pulse that moves all population of the highest-occupied motional state $\ket{g,2}$ into $\ket{e,1}$, keeping track of the effects on other states.
        (c) Apply a carrier pulse to join the electronic states of the elements with the now-largest phonon state into either $\ket{g,1}$ or $\ket{e,1}$.
        (d) Depending on the previous pulse, apply either the red or blue sideband to reduce the highest motional state in the system down by a further phonon.
        (e) Finally, combine the population into the true initial state $\ket{g,0}$.
        The desired creation sequence is the adjoint of the operation just derived.
    }%
\end{figure*}

First consider the operation in reverse, that is with the system starting in the target state of $c_0\ket{g,0} + c_1\ket{g,1} + \dotsb + c_n\ket{g,n}$.
The aim is to ratchet down the highest occupied motional state until only $\ket0$ remains, then combine all population into $\ket{g,0}$ to produce the initial state.
The adjoint of the whole operation will then be the desired creation sequence.
This is illustrated in \cref{fig:coherence-superposition-preparation}.

Explicitly, first apply a red-sideband pulse to move all of the population in $\ket{g,n}$ into the state $\ket{e,\msquash{n-1}}$.
This has an effect on every element in the superposition as well---except for $\ket{g,0}$---that must be tracked.
The different motion levels oscillate with their respective coupled state at generally incommensurate frequencies---approximately $\eta\Omega\sqrt n$ for the red- and blue-sideband transitions.
The total state after the first pulse is some new $c_{g,0}\ket{g,0} + c_{e,0}\ket{e,0} + \dotsb + c_{g,\,n-1}\ket{g,\msquash{n-1}} + c_{e,\,n-1}\ket{e,\msquash{n-1}}$ for complex $\{c\}$, where now various excited electronic states are involved.
The pulse angle and phase to achieve this with a red-sideband pulse exactly on resonance can be derived from \cref{eq:iontrap-sideband-evolution} as
\begin{equation}\label{eq:coherence-creation-angles}
\Omega_{n,\,n-1}t = \pm2\arctan\frac{\abs{c_{g,n}}}{\abs{c_{e,\,n-1}}} \quad\text{and}\quad
\phi = \arg\frac{c_{g,n}}{c_{e,\,n-1}} + \pi,
\end{equation}
and equivalent operations using the carrier and blue-sideband transitions have similar forms, albeit with modified motional levels.
Next combine the populations of the $\ket{g,\msquash{n-1}}$ and $\ket{e,\msquash{n-1}}$ states into one of these two by using the carrier.
Depending on whether the ground or excited electronic state is chosen, apply either the red or blue sideband to reduce the motional level again.
Repeat these steps until the state reaches some $c_{g,0}'\ket{g,0} + c_{e,0}'\ket{e,0}$, then apply a final carrier to reach the initial state $\ket{g,0}$, and take the adjoint of the whole sequence to find the desired forwards mapping.
Creating a motional superposition with a highest occupied phonon number of $n$ requires at most $2n$ pulses.

This algorithm permits a family of solutions.
There is a branch point each time the equal-motion population must be consolidated, since it can be pushed into $\ket g$ or $\ket e$.
The arctangent in \cref{eq:coherence-creation-angles} is also a decision point; since the oscillation frequencies are incommensurate between each pair of coupled states, cycling the upper-most population before consolidating it affects the populations in lower motional states, which may allow for shorter pulses elsewhere in the algorithm.
In motion-changing transitions, more excited motional states generally oscillate faster than lower ones, according to \cref{eq:iontrap-rabi-frequencies,eq:iontrap-laguerre-polynomials}.
For some cases, especially with large superpositions, it may be worth spending longer on a fast pulse in order to save time on slower pulses.
The dominant experimental concerns in choosing the particular solution are minimisation of total pulse time and, to a slightly lesser extent, limiting the number of different transitions addressed.
Most decoherence channel magnitudes are directly related to the total time: motional heating, frequency drift, voltage instabilities, and so forth.
Using fewer transitions reduces experimental complexity by minimising the amount of calibration required.
In theory it should be possible to cleanly address all transitions once the lasers are calibrated to the qubit frequency, and the modulation of the laser is calibrated to the trap frequency.
In practice, though, some components may show non-linear responses to different target frequencies, and the highest-fidelity operations can typically only be achieved after each transition separation frequency and coupling strength are measured individually.

Despite having an infinite number of possible solutions, one can still find the absolute shortest-time solution with a breadth-first search.
First, run the algorithm once making arbitrary decisions about which qubit state to combine populations in, and always taking the shortest pulse angle available as a solution to \cref{eq:coherence-creation-angles}.
The total time taken by this pulse sequence is an upper bound on the time required.
Now, restart the algorithm treating the solution space as a lazily constructed tree (in the computer-science, data-structure sense), where the nodes are decision points and the branches represent particular choices.
Traverse the tree breadth first, attaching the system state (the current statevector and cumulative pulse time taken) to each node as it is encountered.
If a node requires a pulse time greater than the upper bound that was previously found, it is rejected and no further solutions to \cref{eq:coherence-creation-angles} need to be considered at this layer.
This strategy of pruning the tree as it is constructed ensures that the complete, infinite tree never need be built, and the algorithm will eventually terminate having reached a single leaf node that has the minimum possible time.

Our experiment originally only calibrated the carrier and red-sideband transitions, to reduce the complexity of setting up the experiment.
It was found that for the particular motional superpositions chosen, the absolute variation of the optimal superposition-creation sequence from the initial bound---chosen to use only the carrier and red-sideband transitions---was on the order of \qty{1}{\percent} of the total length.
Since this was a very minor improvement, the preference to use only two different transitions generally won out.
The explicit forms of the sequences used in the experimental realisation are given later in \cref{tab:coherence-pulses-12,tab:coherence-pulses-012,tab:coherence-pulses-0123}, along with details of the measurement-mapping sequences that are derived in the next section.


\section{Measurement-mapping sequences}
\label{sec:coherence-mapping}

While \cref{sec:coherence-general-measurements} showed that imperfect measurements cannot produce false positives, one must still choose a suitable mapping sequence $\U_{\text m}$ for a given input state to gain the greatest chance of registering a true positive result.
To illustrate, no matter how faithfully the mapping and measurement $\op A = \proj 00$ is implemented, it is entirely incoherent and its resulting interference pattern will always be constant.
From the numerical work of the previous section, if the input state $\ket\psi$ is an equal superposition of coherence-basis elements, the aim is always to implement a measurement operator $\op A$ that is a rank-1 projector $\proj\psi\psi$.
The standard two-state Ramsey experiment achieves this ideal measurement mapping for its input state $\ket{g,0}+\ket{g,1}$ by a simple inversion of the preparation sequence.
In fact, if the available measurement in a system is a projection onto \emph{exactly} the initial state (before $\op\rho$ is created), inverting the creation sequence is generally sufficient to achieve a high-fidelity mapping for that state.

This simplicity does not extend to the ion-trap measurement $\proj ee\otimes\I_{\text{motion}}$ for general states.
Measurement of the electronic state in the ion trap does not just measure population in the perfect initial state, but also other motional states.
The superposition-creation algorithm described in \cref{sec:coherence-creation} is specifically constructed to map $\ket{g,0}$ to the target state, but its effect on states orthogonal to the target state is not considered at all.
In all probability, the orthogonal states that become populated during the phase evolution will be mapped back to have some population in the ground electronic state, and some in the excited.
The final measurement of the electronic state is therefore not able to completely distinguish the different cases, and the visibility is limited.
Visibility not a direct component of the certifier $R_3$, but a lack of it does qualitatively indicate that the maximal achievable $R_3$ value for this pattern, even implemented perfectly, will be lower than it could be.

While the ion-trap measurement will always fail to distinguish different motional levels, it can be turned into a proxy for a perfect projective measurement by choosing $\U_{\text m}$ to map the target state into $\ket e$, and other states to $\ket g$.
Let us take an experiment to create $\ket{g,0} + \ket{g,1} + \ket{g,2}$ and verify 3-coherence as an explicit example.
We need only consider the space of states that can become populated as a result of the phase evolution.
This leads to a small set of conditions for $\U_{\text m}$:
\begin{equation}\label{eq:coherence-measurement-mapping}\begin{aligned}
\hat{\mathcal U}_{\text m} \bigl(\ket{g,0} + \ket{g,1} + \ket{g,2}\bigr) &\propto \ket{e,\lambda_1},\\
\hat{\mathcal U}_{\text m} \bigl(\ket{g,0} - 2\ket{g,1} + \ket{g,2}\bigr) &\propto \ket{g,\lambda_2},\text{ and}\\
\hat{\mathcal U}_{\text m} \bigl(\ket{g,0} - \ket{g,2}\bigr) &\propto \ket{g,\lambda_3}.
\end{aligned}\end{equation}
States with more than two phonons do not need to be considered in this mapping, since they do not become populated as part of the state creation or evolution.
The choice of the particular other orthogonal states is unimportant, provided the states chosen span the same space as $\{\ket{g,0},\,\ket{g,1},\,\ket{g,2}\}$; by linearity of $\U_{\text m}$, any state in this space orthogonal to the target is a linear combination of the two other states chosen, and so will also be purely in the electronic ground state.

Any choice of the motional states $\{\ket{\lambda_j}\}$---no matter their amount of coherence---is suitable, and equally efficient if the mapping is implemented ideally; the motion is completely traced out and plays no further part.
We still must attempt to minimise the total time of this pulse sequence, though, to avoid the effects of the same decoherence processes described in \cref{sec:coherence-creation}.
Frequency and phases drifts will cause the manipulations to be imperfectly applied, with longer experimental times leading to a higher likelihood of failing to create or verify the coherence.
Motional heating from the trap electrodes or from sympathetic heating from other motional modes cause the phonon numbers to spread out during the operation, naturally destroying the coherence.
As with entanglement witnesses, the more these processes occur, and the greater the distance of the actual created state from the state the mapping sequence was designed for, the less concrete information is likely to be obtained from the method.

The coupled conditions of \cref{eq:coherence-measurement-mapping} do not appear to permit an analytic solution.
Suitable sequences can still be found by numerical methods.
The scheme here is far simpler than the complicated parametrisation used in \cref{sec:coherence-general-measurements} to find the threshold values of $R_3$.
We start with a list of possible sequences of transitions, such as:
\begin{itemize}[topsep=0.2\baselineskip,itemsep=0pt]
\item \textcolor{redsb}{red}, \textcolor{carrier}{carrier}, \textcolor{redsb}{red}, \textcolor{carrier}{carrier}, \textcolor{redsb}{red};
\item \textcolor{carrier}{carrier}, \textcolor{redsb}{red}, \textcolor{bluesb}{blue}, \textcolor{carrier}{carrier}, \textcolor{bluesb}{blue};
\item \textcolor{bluesb}{blue}, \textcolor{redsb}{red}, \textcolor{carrier}{carrier}, \textcolor{bluesb}{blue}, \textcolor{redsb}{red}.
\end{itemize}
Each possibility is optimised for separately.
The choices can be of different lengths; empirically, it seems that for a good mapping one requires at least one more pulse than the corresponding state-creation sequence, and these typically alternate between a sideband and the carrier.

For each pulse sequence, each contained pulse is parametrised by two values: the length of time it is applied for, and its relative phase offset.
An appropriate loss function is the average population that will end up in the opposite electronic state to where it should.
Explicitly, for a target motional superposition state $\ket\psi$ and its orthogonal states $\{\ket{\chi_j}\}$, the minimisation problem is
\begin{equation}\label{eq:coherence-mapping-loss}
\min_{\vec t,\,\vec\phi} \Tr\Bigl[ \bigl(\proj gg\otimes\I_{\text{motion}}\bigr) \U_{\text m}^{}\proj\psi\psi\U_{\text m}^{\smash\dagger} + \sum_j\bigl(\proj ee\otimes\I_{\text{motion}}\bigr) \U_{\text m}^{}\proj{\chi_j}{\chi_j}\U_{\text m}^{\smash\dagger}\Bigr],
\end{equation}
where
\begin{equation}\label{eq:coherence-mapping-subsequence}
\U_{\text m}(\vec t,\,\vec\phi) = \U_{\text m,n}(t_n,\phi_n)\dotsm\U_{\text m,2}(t_2,\phi_2)\U_{\text m,1}(t_1,\phi_1),
\end{equation}
and the individual $\{\U_{\text m,j}\}$ are the evolution operators of each transition.

For any realistic experimental realisation, it is not necessary that the loss function exactly reaches zero.
I somewhat arbitrarily used a cut-off of requiring the outside-target-state probability to be less than \num{e-10}, and counted any optimisation result within this bound as a success.
Each considered sequence of transitions was minimised with a quasi-Newton method equipped with an analytic calculation of the Jacobian of the loss function.\footnote{%
    This is easily, if tediously, calculated from \cref{eq:coherence-mapping-loss,eq:coherence-mapping-subsequence} using derivatives of \cref{eq:iontrap-sideband-evolution}.
}
Separate runs were started from random initial parameter vectors, repeating the process for about three hours per sequence.
For each target state considered in this work, there were many possible sequences with the same number of transitions used, and within each, several possible values of the parameter vector that qualified as a success.
The \emph{best} sequence depends, as before, on the particulars of the experiment, but in general it is sensible to choose the sequences with the least total duration, and potentially only use the carrier and one other sideband to reduce calibration requirements.

The exact sequences used in our experimental realisation of this certifier are given in \cref{tab:coherence-pulses-12,tab:coherence-pulses-012,tab:coherence-pulses-0123}, along with the parameters of the state creation.
These are also available in a more machine-readable format~\cite{Corfield2021Code}.
In the tables, the \emph{pulse length} is the time apply the pulse, scaled such that a value of $1$ is the time taken to completely exchange the populations of the lowest-coupled motional levels in the transition: $\ket{g,0}\leftrightarrow\ket{e,0}$ for the carrier, $\ket{e,0}\leftrightarrow\ket{g,1}$ for the red sideband and $\ket{g,0}\leftrightarrow\ket{e,1}$ for the blue sideband.
Note that the pulse lengths on the carrier have significantly less effect on the total duration than those on a sideband, since the power of sideband transitions is suppressed by a factor of the Lamb--Dicke parameter, typically held at around \sfrac1{10}.
The \emph{phase offset} has the same meaning as it does in \cref{eq:iontrap-sideband-evolution}, namely that the driving field phase be offset by this amount relative to where it would have been had it oscillated freely on resonance since the start of the experiment.
The phase offsets are not cumulative; each is relative to the start of the experiment.

\begin{table*}[!p]%
    \newcolumntype{x}{D..{1.2}}%
    \begin{tabular*}\textwidth{@{\extracolsep{\fill}} l @{\hskip 1.5em} xxxxc @{\hskip 1.5em} xxxxx @{}}\toprule
    & \multicolumn{4}c{State creation} && \multicolumn{5}c{Measurement mapping}\\\cmidrule(lr){2-5}\cmidrule(){7-11}
        Transition & \multicolumn1c{\textcolor{carrier}{carrier}} & \multicolumn1c{\textcolor{redsb}{red}} & \multicolumn1c{\textcolor{carrier}{carrier}} & \multicolumn1c{\textcolor{redsb}{red}} && \multicolumn1c{\textcolor{redsb}{red}} & \multicolumn1c{\textcolor{carrier}{carrier}} & \multicolumn1c{\textcolor{redsb}{red}} & \multicolumn1c{\textcolor{carrier}{carrier}} & \multicolumn1c{\textcolor{redsb}{red}}\\
        Pulse length          &  0.60 &  0.80 &  0.74 &  0.71 &&  0.71 &  0.44 &  1.41 &  0.54 &  1.41\\
        Phase offset ${}/\pi$ &  0    & -0.50 &  0    & -0.50 &&  0    & -0.66 & -0.83 & -0.87 & -0.41\\\bottomrule
    \end{tabular*}%
    \caption[Creation and mapping sequences for $\bigl(\ket{g,0}+\ket{g,1}\bigr)/\sqrt2$]{\label{tab:coherence-pulses-12}%
        Pulse sequences for creation and measurement mapping of target state $\bigl(\ket{g,1} + \ket{g,2}\bigr)/\sqrt{2}$.
        This is the first non-trivial two-element motional state; superpositions of $\ket0$ and either $\ket1$ or $\ket2$ can be created with only two pulses, and an optimal mapping sequence is just the inverse of the creation.
        The state here requires a stricter mapping sequence to produce a full-visibility interference pattern.
        The meanings of the rows are explained in more detail in \cref{tab:coherence-pulses-012}.
    }%
\end{table*}\begin{table*}[!p]%
    \newcolumntype{x}{D..{1.2}}%
    \begin{tabular*}\textwidth{@{\extracolsep\fill}l @{\hskip 1.5em} xxxxc @{\hskip 1.5em} xxxxx @{}}\toprule
    & \multicolumn{4}c{State creation} && \multicolumn{5}c{Measurement mapping}\\\cmidrule(lr){2-5}\cmidrule(){7-11}
        & \multicolumn1c{\textcolor{carrier}{carrier}} & \multicolumn1c{\textcolor{redsb}{red}} & \multicolumn1c{\textcolor{carrier}{carrier}} & \multicolumn1c{\textcolor{redsb}{red}} && \multicolumn1c{\textcolor{redsb}{red}} & \multicolumn1c{\textcolor{carrier}{carrier}} & \multicolumn1c{\textcolor{redsb}{red}} & \multicolumn1c{\textcolor{carrier}{carrier}} & \multicolumn1c{\textcolor{redsb}{red}}\\
        Pulse length          &  0.50 &  0.70 &  0.73 &  0.71 &&  0.71 &  0.48 &  1.42 &  1.58 &  0.71\\
        Phase offset ${}/\pi$ &  0    & -0.50 &  1.00 &  0.50 &&  0    & -0.28 & -0.25 & -0.86 & -0.47\\\bottomrule
    \end{tabular*}%
    \caption[Creation and mapping sequences for $\bigl(\ket{g,0}+\ket{g,1}+\ket{g,2}\bigr)/\sqrt3$]{\label{tab:coherence-pulses-012}%
        Pulse sequence for creation and measurement mapping of target state $\bigl(\ket{g,0} + \ket{g,1} + \ket{g,2}\bigr)/\sqrt3$.
        The pulse length is the duration of the pulse, scaled such that a value of $1$ would completely exchange the populations of the coupled pair of states with the lowest motional occupation.
        The given phase is applied as an offset relative to where the driving field would have been, had it been oscillating freely since the beginning of the experiment.
    }%
\end{table*}\begin{table*}[!p]%
    \newcolumntype{x}{D..{1.2}}%
    \begin{tabular*}\textwidth{@{} l @{\hskip 1.5em\extracolsep{\fill}} xxxxxx @{}}\toprule
    & \multicolumn{6}c{State creation} \\\cmidrule(l){2-7}
        Transition & \multicolumn1c{\textcolor{carrier}{carrier}} & \multicolumn1c{\textcolor{redsb}{red}} & \multicolumn1c{\textcolor{carrier}{carrier}} & \multicolumn1c{\textcolor{redsb}{red}} & \multicolumn1c{\textcolor{carrier}{carrier}} & \multicolumn1c{\textcolor{redsb}{red}}\\
        Pulse length          &  0.51 &  0.55 &  0.96 &  0.57 &  0.84 &  0.58\\
        Phase offset ${}/\pi$ &  0    & -0.50 & -1.00 &  0.50 &  0    & -0.50\\
    \end{tabular*}\\%
    \begin{tabular*}\textwidth{@{} l @{\hskip 1.5em\extracolsep{\fill}}xxxxxxxxx @{}}\midrule
    & \multicolumn{9}c{Measurement mapping}\\\cmidrule(){2-10}
        Transition & \multicolumn1c{\textcolor{redsb}{red}} & \multicolumn1c{\textcolor{carrier}{carrier}} & \multicolumn1c{\textcolor{bluesb}{blue}} & \multicolumn1c{\textcolor{carrier}{carrier}} & \multicolumn1c{\textcolor{redsb}{red}} & \multicolumn1c{\textcolor{carrier}{carrier}} & \multicolumn1c{\textcolor{redsb}{red}} & \multicolumn1c{\textcolor{carrier}{carrier}} & \multicolumn1c{\textcolor{redsb}{red}}\\
        Pulse length          &  2.89 &  1.47 &  1.15 &  3.02 &  2.31 &  4.69 &  2.31 &  0.72 &  0.58\\
        Phase offset ${}/\pi$ &  0    & -0.16 & -0.41 & -0.53 &  0.45 &  0.79 & -0.32 & -0.13 &  0.76\\\bottomrule
    \end{tabular*}%
    \caption[Creation and mapping sequences for $\bigl(\ket{g,0}+\ket{g,1}+\ket{g,2}+\ket{g,3}\bigr)/2$]{\label{tab:coherence-pulses-0123}%
        Pulse sequences for creation and measurement mapping of target state $\bigl(\ket{g,0} + \ket{g,1} + \ket{g,2} + \ket{g,3}\bigr)/2$.
        This four-element superposition state involved optimising over 18 parameters in an optimisation landscape with many loss-function minima.
        The long durations of some of the pulses in the measurement-mapping component suggest that there may have been solutions with less time requirements to be found, had we had more time to run optimisations on the compute clusters.
        The meanings of the rows are explained in more detail in \cref{tab:coherence-pulses-012}.
    }%
\end{table*}

We were not able to determine a rigorous structure to the pulse sequences found, but there are some features of note.
The optimiser strongly preferred sequences that alternate between applying a sideband and applying the carrier.
Its first pulse was very frequently the inverse of the last pulse in the state-preparation sequence, which reduces the maximum number of phonons in the system by one.
This pulse also populates the majority of the states with lower motional occupation and both a ground-state and excited-state ion.
Qualitatively, this allows subsequent pulses to effect more interactions because different states.
When the optimiser uses sideband pulses after the initial drive, it often seems to choose a length that would cause the largest motional state to do a complete population cycle with the greater unpopulated state it is coupled to.
For example, if the highest motional level at an intermediate point was $\ket{e,1}$, as in \cref{tab:coherence-pulses-12,tab:coherence-pulses-012}, the applied red sideband would tend to have a duration of $\sqrt2\approx1.41$, which transfers the population of $\ket{e,1}$ into $\ket{g,2}$ and then back again, leaving the total phonon number unchanged.
The other coupled states do not share the same period, so this appears to be a method by which the optimiser modifies the relative populations in each level, without increasing the maximal excitation.

Beyond these observations, the structure of the optimised pulse sequences appears relatively opaque.
The actual motional states that the optimiser maps the different elements to do not have any clear significance.
As an example, when performing the measurement mapping for the state $\ket{g,0} + \ket{g,1} + \ket{g,2}$ using the sequence given in \cref{tab:coherence-pulses-012}, the different motional superpositions $\{\ket{\lambda_j}\}$ in \cref{eq:coherence-measurement-mapping} are
\begin{equation}\begin{aligned}
\ket{\lambda_1} &\approx \ket0,\\
\ket{\lambda_2} &\approx \msquash{(0.43 + 0.11i)}\ket0 + 0.75\ket1 + \msquash{(0.44-0.23i)}\ket2,\ \text{and}\\
\ket{\lambda_3} &\approx 0.72\ket0 - \msquash{(0.26+0.25i)}\ket1 + \msquash{(0.05+0.59i)}\ket2,
\end{aligned}\end{equation}
up to global-phase equivalence.
It is interesting that the target state is mapped back to the initial motional state, especially because the complete state is \emph{not} the initial state, but is instead $\ket{e,0}$.
For the purposes of certification of coherence, however, we do not need to understand the inner machinations of the optimiser, but simply to use its mappings to implement a perfect projective measurement.


\section{Statistics of the certifier}

One final, major obstacle to the robustness of the $R_3$ certifier is the impact of imperfect measurement statistics on the interference pattern.
So far, the theory work has assumed that the value of $R_3$ can be determined accurately, and within this assumption it cannot return a false positive for higher-order coherence.
This does not survive contact with the real world, however.
When measuring an interference pattern, one samples each point by taking a series of \emph{click/no-click} shots and using their count to estimate the probability of an underlying binomial distribution.
This provides two mechanisms by which an interference pattern can appear to exhibit coherence that is not present: one in the sampling statistics of individual distributions, the other in the approximation of the continuous interference pattern by a series of discrete points.

As an extreme example, consider the case the state $\op\rho = \proj00 + \proj11$.
This is completely incoherent and its interference pattern should be a constant value of \sfrac12, but misfortune in the binomial sampling could show false evidence of oscillation.
One must account for the standard errors of the measurements when calculating the uncertainty in the value of $R_3$, but the ratio of moments of the interference pattern is highly non-linear, meaning the standard physicists' workhorse
\begin{equation}\label{eq:coherence-error-propagation}
\alpha_f \approx \sqrt{\sum_j\abs*{\frac{\partial f}{\partial x_j}}^2\alpha_{x_j}^2},
\end{equation}
is not valid.
Further, the non-linearity introduces a systematic upwards bias in the most natural estimators of $R_3$, which must be corrected.

We must first begin with the basics: accurately assessing the uncertainty in individual points of the interference pattern.
Each datum is modelled by a binomial distribution with some exactly underlying probability $\mu_j$ and a number of shots $n$ taken.
The probability of excitation of point $j$ is then a random variate $P_j\sim\operatorname B(n,\mu_j)/n$ from which our estimates $p_j$ are drawn.
The unbiased maximum-likelihood estimator of $\mu_j$ is very naturally to define $p_j$ to be the number of successes observed divided by the number of shots.
Continuing this, the first statistical moments of the binomial distribution have unbiased estimators (symbols with hats)~\cite{Chan2020}
\begin{equation}\label{eq:coherence-statistical-moments}\begin{alignedat}3
\expect[\big]{{(P_j - \mu_j)}^1} &\to p_j - \hat\mu_j &&= 0 &\quad&\text{(mean)}\\
\expect[\big]{{(P_j - \mu_j)}^2} &\to \hat \sigma_j^2 &&= \frac{p_j (1 - p_j)}{n-1} &&\text{(variance)}\\
\expect[\big]{{(P_j - \mu_j)}^3} &\to \hat \kappa_j &&= \frac{p_j (1 - p_j) (1 - 2p_j)}{(n-1)(n-2)} &&\text{(skewness)},
\end{alignedat}\end{equation}
where $\expect{X}$ is the expectation of a random variable $X$.

One should not mistake an estimator of the standard deviation of a non-normal statistical distribution for the best estimator of one's confidence in its mean, however.
This is commonly done for the binomial distribution, leading to a zero-width interval if zero successes or failures are measured in the finite number of samples, which is in part necessary to avoid a confidence interval that includes invalid values.
A more appropriate estimation is the Wilson score interval~\cite{Wilson1927}.
This is also based on a normal approximation, but using a slightly different approach, given a number of observed successes $k$ and some desired $z$ score.
The standard estimator answers the question \emph{if the distribution has a mean of $p_j$, what is the expected variance?} while the Wilson score answers \emph{for what range of $\mu_j\mkern-2mu$ would $k\mkern-2mu$ successes be within the expected interval?}
Explicitly, the bounds---without a continuity correction---are the solutions to the quadratic equation
\begin{equation}
\biggl(1 + \frac{z^2}{n}\biggr) p_{j,\text w}^2 - \biggl(2p_j + \frac{z^2}{n}\biggr) p_{j,\text w} + \bigl(p_j^2\bigr) = 0,
\end{equation}
where the quantities in brackets are known.
This interval is not centred on $p_j$, but is asymmetric; this represents the greater variance that binomial variables with mean close to \sfrac12 possess.
All results involving binomial variables presented in this thesis use this method for estimating the uncertainty.

We now move to the larger question of estimating $R_3$ from real experimental data.
The integrals in the measurement must be discretised.
They vary sufficiently smoothly that a trapezium rule over the $J$ different points in the pattern is appropriate.
This is
\begin{equation}
\frac1{2\pi}\int_0^{2\pi} f(x)\,dx \approx \sum_{j=0}^{J-1} w_j f\Bigl(\frac{2\pi j}{J-1}\Bigr),
\quad\text{where\ }
w_j = \begin{cases}\frac1{2(J-1)}&\text{$j=0$ or $j=J-1$}\\\frac1{J-1}&\text{all other $j$,}\end{cases}
\end{equation}
leading to the observed values of $R_3$ being drawn from a random variable
\begin{equation}
R_3 \sim \frac{\sum_j w_j P_j^3}{\bigl(\sum_j w_j P_j\bigr)^2}.
\end{equation}
As alluded to previously, the expectation of this estimator is systematically biased away from the \emph{true} value that would be obtained if the $\mu_j$ were known with certainty.

We can analytically evaluate the expectation $\expect{R_3}$ to determine its bias.
The expectation from each point is $\expect{P_j} = \expect[\big]{\mu_j + (P_j - \mu_j)}$.
This unusual form is to permit a Taylor expansion around the binomial means in terms of $\expect[\big]{(P_j-\mu_j)^n}$, which have known unbiased estimators in \cref{eq:coherence-statistical-moments}.
The expansion proceeds, up to terms of third order, as
\begin{equation}\label{eq:coherence-expectation-r3}\begin{alignedat}3
    \expect{R_3} &= \expect[\bigg]{\Bigl(\sum_j w_j P_j^3\Bigr){\Bigl(\sum_j w_j P_j\Bigr)}^{-2}}\span\span\span\span\\[2.5\jot]
    &\approx \frac{\tilde m_3}{\tilde m_1^2}
        +{}&&\frac1{\tilde m_1^2}\sum_j\Biggl(&&
            3w_j\Bigl(\mu_j - \frac2{\tilde m_1}\mu_j^2 + \frac{\tilde m_3}{\tilde m_1^2}\Bigr)\expect[\big]{{(P_j - \mu_j)}^2}\\[-3\jot]
            &&&\zsaveposx{expectation-r3-left}\makebox[0pt][l]{%
                $\displaystyle\underbrace{\vphantom{\Biggr)}\rule{\dimexpr\zposx{expectation-r3-right}sp-\zposx{expectation-r3-left}sp\relax}{0pt}}_{\text{bias term}}$}%
            &&{} + w_j\Bigl(1 - \frac6{\tilde m_1^2}w_j\mu_j + \frac9{\tilde m_1^2}w_j^2\mu_j^2 - 4\frac{\tilde m_3}{\tilde m_1^3}w_j^2\Bigr)\expect[\big]{{(P_j - \mu_j)}^3}
    \Biggr)\zsaveposx{expectation-r3-right},
\end{alignedat}\end{equation}
where $\tilde m_1 = \sum_j w_j \mu_j$ and $\tilde m_3 = \sum_j w_j \mu_j^3$.
The actual target of the estimator is the quantity $\tilde m_3/\tilde m_1^2$.
We align our estimator of $R_3$ to the centre of the distribution by subtracting the marked bias term.

This new estimator is now \emph{fair}, in that it is as likely to return a value that is too great as it is to return one that is too small.
We still must calculate our estimate of the uncertainty in the value of $R_3$, however.
In this case, I found from a multitude of Monte-Carlo testing that the low-order \emph{propagation of uncertainty} formula of \cref{eq:coherence-error-propagation} produces perfectly acceptable results.
This is somewhat expected; direct measurements of $R_3$ appear to be well approximated by a normal distribution, and there is no covariance between the separate $p_j$.
We must then evaluate the partial derivatives.
Writing the unbiased estimator as $R_{3,\,\text{est}} = R_{3,\,\text{direct}} - \sum_j z_{2,j} - \sum_j z_{3,j}$, where the $\{z_{n,j}\}$ are the terms in \cref{eq:coherence-expectation-r3} that include $\expect[\big]{(P_j - \mu_j)^n}$, the derivatives are
\begin{equation}\begin{aligned}
\frac{\partial z_{2,k}}{\partial p_j} ={}
&\frac{p_k(1-p_k)}{n-1} \Biggl(
   \frac{8w_jw_k^2p_k^2}{\tilde m_1^4}
    + \frac{9w_jw_k^2p_j^2}{\tilde m_1^4}
    - \frac{6w_jw_kp_k}{\tilde m_1^3}
    - \frac{12w_jw_k^2\tilde m_3}{\tilde m_1^5}
\Biggr)\\
&{}+\delta_{jk} \Biggl[
    \frac{p_k(1-p_k)}{n-1}\biggl(
        \frac{3w_k}{\tilde m_1^2} - \frac{12w_k^2p_k}{\tilde m_1^3}
    \biggr)
    + \frac{1-2p_k}{n-1}\frac{3w_k}{\tilde m_1^2}\biggl(
        p_k - \frac{2w_kp_k^2}{\tilde m_1} + \frac{\tilde m_3w_k}{\tilde m_1^2}
    \biggr)
\Biggr]
\end{aligned}\end{equation}
for the second-order correction terms, and
\begin{equation}\begin{alignedat}2
\frac{\partial z_{3,k}}{\partial p_j} = {}
&\frac{p_k(1-p_k)(1-2p_k)w_j}{(n-1)(n-2)} \Biggl(
    - \frac{2w_k}{\tilde m_1^3}
    + \frac{20\tilde m_3w_k^3}{\tilde m_1^6}
    + \frac{24w_k^2p_k}{\tilde m_1^4}
    - \frac{36w_k^3p_k^2}{\tilde m_1^5}
    - \frac{12w_k^3p_j^2}{\tilde m_1^3}
\Biggr)\span\span\\[\jot]
&{}+\delta_{jk} \frac{w_k}{\tilde m_1^2}\Biggl[
    &&\biggl(
        \frac{18w_k^2p_k}{\tilde m_1^2}
        - \frac{6w_k}{\tilde m_1^2}
    \biggr)\frac{p_k(1-p_k)(1-2p_k)}{(n-1)(n-2)}\\[-0.5\jot]
    &&&{}+\biggl(
        1
        - \frac{6w_kp_k}{\tilde m_1^2}
        + \frac{9w_k^2p_k^2}{\tilde m_1^2}
        - \frac{4\tilde m_3w_k^2}{\tilde m_1^3}
    \biggr)\frac{6p_k^2 - 6p_k + 1}{(n-1)(n-2)}
\Biggr]
\end{alignedat}\end{equation}
for the third-order corrections.

The validity of these estimators are tested by Monte-Carlo simulation.
We simulate the measurement of an interference pattern generated by an idealised realisation of the superposition state $\ket0+\ket1+\ket2$, including a perfect projective measurement onto this exact target state.
The resulting value of the $R_3$ metric should be exactly \sfrac{47}{27}.
The estimated value and uncertainty in $R_3$ are calculated for one million different attempts, for both the original and unbiased forms of the estimators, each using the same set attempts.
The integral discretisation was done with a 31-point trapezium rule, and each point of the pattern was sampled one hundred times.
From this, one can derive the probability density function (\textsc{pdf}) of the results by binning the measured values into a histogram of suitable resolution.

\begin{figure}%
    \includegraphics{coherence-pdf.pdf}%
    \caption[\textsc{pdf}s of biased and corrected estimators]{\label{fig:coherence-pdf}%
    Probability density functions for the biased and unbiased estimators of $R_3$ using a 31-point trapezium rule with 100 shots per point, scaled to have a maximal value of one.
    The \textsc{pdf}s (crosses) are derived from one million Monte-Carlo simulations of measuring a pattern whose true $R_3$ is \sfrac{47}{27}, with both estimators using the same set of data.
    Each \textsc{pdf} is overlaid on a Gaussian approximation (lines) using the estimated values of mean and uncertainty.
    It is clear that the na\"ive estimator systematically overestimates.%
}%
\end{figure}

The results of these simulations are shown in \cref{fig:coherence-pdf}.
It is clear that the na\"ive estimator has a quantifiable bias, and our modifications remove this.
Each \textsc{pdf} is shown in comparison to a normal distribution, centred on the mean of the measured values of $R_3$, showing the validity of the normal approximation to the uncertainty calculation.
On very close inspection, there appears to be a very small skewness in the distribution of the estimated values, some of which is to be expected from the truncation of the Taylor expansion when creating the estimator.

With the exception of trifling details such as \emph{a functioning, controllable ion trap}, we now have all the ingredients in place to create arbitrary motional superpositions, evolve them, and verify that we successfully created higher-order coherence.


\section{Experimental realisation}

Over the course of my studies, the experimental ion-trapping group at Imperial built a new linear radio-frequency trap with a single-segment bladed design.
The details of this are given in the thesis of Ollie Corfield~\cite{Corfield2022}---I was not involved in its construction.
This section describes the creation of several-element superpositions in the motional state of a single trapped ion, and their subsequent verification of coherence rank using the methods of the previous sections.
I was not part of the lab team implementing the experiment, but I was responsible for interpreting the data; I performed the fitting of the blue-sideband Rabi experiment and of the most likely 3-coherent state to the final results.

The particular ion was \ce{^{40}Ca+}, with the two qubit states encoded as an optical qubit with $\ket g = 4^2S_{\frac12,\,m_j=-\frac12}$ and $\ket e = 3^2D_{\frac52,\,m_j=-\frac12}$.
This was addressed by a single sub-kHz-linewidth diode laser with a wavelength of approximately \qty{729}{\nano\m} giving a Rabi frequency $\Omega\approx\qty{90}{\kilo\Hz}\cdot2\pi$ on the carrier, and readout by the fluorescence measurement shown in \cref{fig:iontrap-ca40}.
The initial-state ($\ket{g,0}$) preparation sequence has a probability of success of \qty{98(2)}{\percent} through Doppler and then sideband cooling, while the measurement fidelity is reliably above \qty{99}{\percent}.
In terms of the system Hamiltonian from \cref{eq:iontrap-lab-hamiltonian}, the qubit-separation frequency is $\omega_{eg}\approx\qty{411}{\tera\Hz}\cdot2\pi$ and the motional frequency is $\omega_z\approx\qty{1.1}{\mega\Hz}\cdot2\pi$, giving a Lamb--Dicke parameter $\eta \approx 0.09$.
This is well within the Lamb--Dicke regime, and as such the higher-order sidebands are not reasonably available.
With the sideband transitions being suppressed by around an order of magnitude compared to the carrier, there is a sizeable \textsc{ac} Stark effect shifting the frequencies of these transitions.
To mitigate this to first order, an additional compensation pulse is applied far off-resonantly, halfway between the carrier and the opposite-colour sideband.

\subsection{State creation}
\label{sec:coherence-experiment-state-creation}

We attempted to create three different motional superpositions, and verify the coherence rank of each.
The three states were $\ket1+\ket2$ for a 2-coherent state, then $\ket0+\ket1+\ket2$ for a 3-coherent state and $\ket0+\ket1+\ket2+\ket3$ for 4-coherence.
For 2-coherence, $\ket1+\ket2$ was chosen as it is the first state that requires the non-trivial state-creation and measurement-mapping sequences to generate it.
Note that the relative phases of the superposition elements are important in both parts of the process.
We kept all elements the same phase for simplicity in describing them.

After the state creation, one can take an incoherent measure of the populations of each state as a sanity check.
All population should be in the electronic ground state of the ion, and this can be tested directly with the standard ion-trap measurement.
Due to the nature of the superposition-creation algorithm, having all population in $\ket g$ is a very good indication that the algorithm was successful; miscalibrated frequencies, pulse powers or phase drift would all cause appreciable population to remain in $\ket e$ at the completion of the algorithm.

We can also make an estimate of the populations in each motional level with a Rabi-type experiment.
The superposition state is evolved by applying the blue sideband for a varying amount of time.
Each motional level $\ket{g,n}$ will undergo oscillation between the ground and excited electronic states at a rate proportional to $\sqrt{n+1}$ inside the Lamb--Dicke regime.
Scanning the length of the pulse builds up a pattern such as \cref{fig:coherence-bsb} for the three-element superposition.

\begin{figure*}%
    \includegraphics{coherence-bsb.pdf}%
    \caption[Dynamics of the motional blue sideband transition]{\label{fig:coherence-bsb}%
    Excited-state population of state $\ket{g,0} + \ket{g,1} + \ket{g,2}$ while being driven by the blue sideband.
    The data points (purple crosses) are shown with their Wilson binomial 1-$\sigma$ confidence interval.
    The pattern from a state that best fit the data (blue line) was found by maximum-likelihood estimation, and its 95\% confidence region (blue shaded region) by bootstrapping the measured data \num{14000} times.
    The fit accounted for the Rabi frequency, detuning, motional dephasing rate, basis-state populations up to $\ket3$ inclusively, and correlations between directly coupled elements.
    The populations in $\ket{g,0}$, $\ket{g,1}$ and $\ket{g,2}$ were \SI{33(2)}{\percent}, \SI{30(2)}{\percent} and \SI{33(2)}{\percent}, respectively, with \SI{4.7(14)}{\percent} elsewhere.
    All appreciable undesired population was in the $\ket e$ excited qubit state; the motional state $\ket3$ was included in the fit, but found to have a population consistent with zero with a standard error of \num{9e-3} percentage points.
}
\end{figure*}

One can approximately determine the populations of the different basis states from this pattern.
Ideally, we would take a Fourier transform of the data and examine the amplitudes of the different components to find the populations.
This is difficult in practice, however.
Since the component oscillation frequencies are 1, $\sqrt2$, $\sqrt3$, and so on, any discrete Fourier transform will not have bins centred on these values, and there is significant leakage between the frequencies unless the scan can last far longer.
The coherence time of the motion, while not directly measured, limits this scan from being taken much beyond the \qty{1}{\milli\s} shown in \cref{fig:coherence-bsb}.

Instead, we can estimate the populations by a maximum-likelihood method.
In essence, one parametrises a model, then finds the values for which the resulting interference pattern would have the greatest probability of returning the observed data.
This likelihood is the product
\begin{equation}
\ell(\vec x) = \prod_j \Pr\bigl[\operatorname B\bigl(n, p_j(\vec x)\bigr) = n_j\bigr],
\end{equation}
where $n_j$ is the number of times the measurement returned $\ket e$ at a given point, and $p_j$ is the simulated probability at that point for a parameter vector $\vec x$.
The maximum of this cannot generally be calculated analytically.
For numeric evaluation, it is generally advisable to use the log-likelihood $\ln(\ell)$ to avoid floating-point underflow.
Doing so has no effect on the location of any maxima as the natural logarithm is a monotonically strictly increasing function over this domain.
Any standard optimisation routine can then be used to maximise the quantity.

The parameters to be fit to the data here were: the base Rabi frequency of the sideband; a detuning from the actual transition frequency; the basis-state populations of $\ket{g,n}$ and $\ket{e,n}$, with $n$ up to one phonon larger than the maximum expected; and phase correlations between pairs of states coupled by the blue sideband.
In addition, we account for potential motional dephasing by simulating the Lindblad master equation of \cref{eq:qi-lindblad-master} with an additional jump operator $\op L = \sqrt\gamma \a^\dagger\a$ for some rate $\gamma$ included in the fit.
Physically this additional term is a proxy for the effect of a drifting laser phase.
We satisfy the requirement that this operator is bounded by truncating the Hilbert space considered to only the required motional levels.
There are no processes being modelled that would cause the population to move outside this subspace.

Performing maximum-likelihood estimation alone does not give any indication of the level of confidence in the fitted values.
The simplest method for obtaining an estimate of this is by bootstrapping.
In this, the sample of measured data points are repeatedly \emph{resampled} to gain knowledge of how the fit responds to variations in the sampled values.
For each new sample, the same maximum-likelihood estimation is performed many times, to give an estimate of the extent of the uncertainty.
There are multiple possible methods for performing the resampling, depending on the type of problem.
The most straightforward is simply to select a new set of $n$ points from the measured data with replacement, such that some points are sampled more than once and some not at all.
Alternatively, one can consider a \emph{parametrised} resampling, where the maximum-likelihood fit is taken to be the underlying pattern, and each new set of data is generated by drawing each point from the binomial distribution implied by the pattern.
In the analysis here, I used both of these methods for 180 \textsc{cpu}-hours per method per state, which resulted in approximately \num{14000} realisations for each.
The \qty{95}{\percent} confidence intervals for the true pattern from the two methods, as shown by the light region in \cref{fig:coherence-bsb} for $\ket{g,0}+\ket{g,1}+\ket{g,2}$, were within statistical noise of each other.

\begin{table*}%
    \begin{center}\begin{tabular*}{0.8\linewidth}{@{\extracolsep{\fill}}l@{\hspace*{4em}}ccc@{}}\toprule%
    Target state in $\ket g$& $\eta\Omega/(2\pi\,\si{\kilo\Hz})$ & $\delta /(2\pi\,\si{\kilo\Hz})$ & $\gamma/(2\pi\,\si{\kilo\Hz})$ \\\midrule%
    $\phantom{\ket0+{}}\ket1+\ket2$ & \num{8.39(3)} & \num{1.3(5)} & \num{0.11(2)}\\%
    $\ket0+\ket1+\ket2$ & \num{6.95(3)} & \num{2.6(2)} & \num{0.14(3)}\\%
    $\ket0+\ket1+\ket2+\ket3$ & \num{7.52(2)} & \num{1.5(3)} & \num{0.22(4)}\\%
    \bottomrule\end{tabular*}\end{center}\vspace*{-\baselineskip}%
    \caption[Fit parameters after superposition creation]{\label{tab:coherence-fit-parameters}%
        Values and uncertainties in the fit parameters after bootstrapping the measured data for each of the three target states.
        The modified Rabi frequency $\eta\Omega$ is the coupling strength to the lowest motional state on the blue sideband at zero detuning.
        Each detuning $\delta$ is on the order of \sfrac1{500} of the sideband separation frequency, but closer in size to the Rabi frequency than is desirable.
        The dephasing strength $\gamma$ is relatively small compared to all other parameters.
    }%
\end{table*}

\begin{table*}%
    \begin{tabular*}{\linewidth}{@{\extracolsep{\fill}}l@{\hspace*{2em}}lllll@{\hspace*{3em}}l@{}}\toprule%
    Target state in $\ket g$& \multicolumn6c{Estimated population in basis state}\\\cmidrule{2-7}%
    & $\ket{g,0}$ & $\ket{g,1}$ & $\ket{g,2}$ & $\ket{g,3}$ & $\ket{g,4}$ & $\ket e$\\\midrule%
    $\phantom{\ket0+{}}\ket1+\ket2$ & \num{0.052(14)} & \num{0.546(14)} & \num{0.383(13)} & \num{0.000(5)} & & \num{0.018(9)}\\%
    $\ket0+\ket1+\ket2$ & \num{0.33(2)} & \num{0.31(2)} & \num{0.33(2)} & \num{0.000(12)} & & \num{0.036(11)}\\%
    $\ket0+\ket1+\ket2+\ket3$ & \num{0.29(2)} & \num{0.25(2)} & \num{0.21(2)} & \num{0.223(15)} & \num{0.000(11)} & \num{0.026(11)}\\%
    \bottomrule\end{tabular*}%
    \caption[Fitted motional populations after superposition creation]{\label{tab:coherence-fit-populations}%
        State populations, not complex amplitudes, for individual quanta of motion in the ground qubit state, and the total population in the excited qubit state.
        The values and their uncertainties were found by maximum-likelihood estimation while bootstrapping the measured data.
        Each state shows some deviation from the ideal, but not excessive.
        In all cases, the population in motional states larger than those in the superposition was statistically consistent with zero.
    }%
\end{table*}

The measured parameters for the three states are shown in \cref{tab:coherence-fit-parameters,tab:coherence-fit-populations}.
While the Rabi frequencies are consistent with the measured values from calibration, the detunings exceed what was expected.
It is plausible that the simulation model was partially underparametrised, such as by failing to account for inefficiencies in the measurement process or off-resonant excitations from the \textsc{ac} Stark effect, and that this led to better fits with a larger detuning.
Maximum-likelihood estimation is, after all, only as good as its model.
In this case, the significantly larger uncertainties on the detuning suggest that it had relatively little impact on the model, and there is still good reason to believe the other parameters with much lower relative uncertainties.


\subsection{Coherence certification}

With the motional states created and sanity-checked, all that is left is to attempt to verify the rank of the coherence that has been created.
The phase evolution $\U_{\text f}$ of \cref{eq:coherence-interference-pattern-povm} is implemented virtually in this system; each transition driven after the period of the phase evolution is offset by an amount $n\phi$, where $n$ is the difference in the number of phonons between the coupled ground and excited qubit states: $0$ for the carrier, $-1$ for the red sideband and $1$ for the blue sideband.
This is exactly equivalent to evolution under the free Hamiltonian $\H \propto \sum_j j\proj jj$ if only the motional states are considered.

For the certification scheme, it is vital that no evolution-phase-dependent error enters the measurement-mapping operation $\U_{\text m}$.
This is a core assumption of the derivations in the threshold values of the certifier.
It is possible in theory for a completely incoherent state $\ket0$ to be falsely detected as 2-coherent if the mapping sequence somehow projected it onto the state $\cos\phi\ket0+\sin\phi\ket1$.
While this is an extreme example, we still must consider any ways that our system could introduce any phase-dependent error.

The virtual phase advancement on the surface looks plausible, but since the resulting pattern must be periodic, this would be trivially detectable; if the measured interference patterns did not appear to have a period of $2\pi$, it would be clear that some catastrophic failure of the control systems had occurred.
This phase advancement is applied in the arbitrary-waveform generator in the same manner as all other phase offsets, and consequently any introduced error is independent of the magnitude of the shift.
More detail on how the pulses are actually synthesised is presented in the theses from the experimental group~\cite{Corfield2022}.
In fact, applying the phase advancement virtually has an advantage in that it is a constant-time operation, so there is no chance that run-time-dependent processes will enter as each shot has precisely the same duration.

Beyond this, it is possible that some drifting parameters in the lab could have a time-dependent effect on measurement efficiency or the mapping sequence.
With all the cooling and measurement cycles and at 400 shots per point, the final data collection period for each experiment was around ten minutes.
This was in part limited by the length of time the system calibration was valid for; beyond this, and the trap parameters may have drifted far enough that the fidelity of the coherent manipulations was compromised.
To avoid any possibility that the environment had an effect that appeared to be phase-dependent, the different shots of the experiment were interleaved.
24 different phase-advancement values were taken, and each was repeated 400 times.
Rather than running every shot for one point, and moving on sequentially, a random order of the 24 points was generated, and then this order was cycled through taking one hundred shots per raster.
Limitations in the control system made it impractical to randomise the order for each individual shot.
Regardless, this has the approximate effect of converting any time-dependent noise from the environment into a white-noise process, which is safely handled by the coherence certifier.

A final possibility is for a systematic mis-set of the driving frequencies.
If the qubit frequency is mis-set from the average such that the red- and blue-sideband transitions are addressed at different detunings, any phase advancement on them could in principle introduce a phase-dependent error.
This could arise from an imperfect compensation of the \textsc{ac} Stark effect, or by calibrating both sideband frequencies using only a measurement of the qubit frequency and the trap frequency.
In this experiment, the two sideband locations were calibrated independently, which alleviates the majority of this concern.
The drift of the laser frequency with respect to the sideband frequency over the course of an entire experiment was estimated by pre- and post-calibration to be less than $\delta/\Omega=0.15$, where the time-dependent components of this will have been converted to incoherent noise by the shot randomisation described previously.

\begin{figure*}%
    \includegraphics{coherence-phase-2-3.pdf}%
    \caption[Interference patterns for two- and three-element superpositions]{\label{fig:coherence-phase-2-3}%
    Measured interference patterns (purple crosses) for the two- and three-element motional superpositions indicated, using the measurement-mapping sequences described in \cref{tab:coherence-pulses-12,tab:coherence-pulses-012}.
    Each point was repeated 400 times, and the Wilson 1-$\sigma$ confidence intervals are indicated with error bars though these small enough to be difficult to see.
    The models (green lines) are the theoretically optimal interference patterns, had there been no experimental error or noise.
    The measured certifiers were greater than the level needed to certify 2-coherence (1) and 3-coherence (1.25) respectively.
    }%
\end{figure*}

With all this in hand, \cref{fig:coherence-phase-2-3} shows the measured interference patterns and ideal models for the two target states $\ket{g,1}+\ket{g,2}$ and $\ket{g,0}+\ket{g,1}+\ket{g,2}$.
Of these, the two-element superposition had a measured value of $R_3 = \num{1.090(12)}$ after the bias correction, which is above the threshold of \num{1} officially needed to certify 2-coherence with this metric, though of course the evidence of oscillation alone is sufficient in this case.
As expected, the certifier value does not reach \num{1.25}, the maximum achievable with a 2-coherent state, due to imperfections in the experimental environment.

The three-element superposition, on the other hand, did reach above this, to $R_3 = \num{1.54(2)}$.
Achieving this value certifies that the state created \emph{must} have contained higher-order coherence; to reiterate, it is impossible for a state without genuine 3-coherence to exhibit a value of $R_3$ larger than \num{1.25}.
For this state, the optimised mapping pulses are absolutely necessary to this certification; without them, the certifier would have a theoretical maximal value of $R_3\approx\num{0.92}$.
This is significantly below the threshold needed to certify, in part because the peak-to-peak visibility of the pattern could not have exceeded \num{0.68}.

\begin{figure}%
    \includegraphics{coherence-phase-4.pdf}%
    \caption[Interference patterns of a four-element motional superposition]{\label{fig:coherence-phase-4}%
    As in \cref{fig:coherence-phase-2-3}, but for the four-element superposition indicated with the measurement-mapping sequence described in \cref{tab:coherence-pulses-0123}.
    The value of the certifier here, $R_3 = \num{1.35(3)}$, does not reach the level of $\text{\sfrac{179}{96}} \approx 1.86$ that is necessary to unambiguously certify 4-coherence, but it is at least 3-coherent.
    The interference pattern of the 3-coherent state that best approximates both this and the state-population data is also shown (dashed blue line), where the thinner, lighter line within is its 1-$\sigma$ Wilson interval.
    The unconvincing fit is evidence that the state likely was 4-coherent, but the certifier test was inconclusive.
}%
\end{figure}

Attempting to stretch the trap---which was not specifically designed to have any particular resilience to motional decoherence---to its limits, we also pursued the four-element superposition $\ket{g,0}+\ket{g,1}+\ket{g,2}+\ket{g,3}$.
The resulting interference pattern is shown in \cref{fig:coherence-phase-4}.
This state did not cross the threshold of \num{1.86} necessary to certify 4-coherence, but its value of $R_3 = \num{1.35(3)}$ was still larger than \num{1.25}, unequivocally still certifying it as 3-coherent.

In all three cases, the observed interference patterns were of lower total visibility than they theoretically could have been.
\Cref{fig:coherence-phase-2-3,fig:coherence-phase-4} also show the ideal interference patterns that could have been measured, had the state preparation and measurement mapping been implemented without any experiment error.
For the two-element superposition, this would have been the threshold value \num{1.25}, while for the three-element superposition it would have been $\text{\sfrac{47}{27}}\approx\num{1.74}$.
This is not the absolute maximum achievable for a 3-coherent state, but well above the threshold.
The currently known maximum is \num{1.77}, which is exhibited by a superposition with the central element weighted slightly higher than the outer two~\cite{Dive2020}.
The four-element superposition could potentially have reached $\text{\sfrac{145}{64}}\approx\num{2.27}$.

It is natural that the highest possible values of $R_3$ were not observed in this real experiment.
The fits of the state populations in \cref{tab:coherence-fit-populations} show that it is likely the state creation was not exact, and one naturally expects that the measurement-mapping operations, being slightly more complex, would similarly be implemented with some inaccuracies.
Frequency drift, off-resonant excitation and thermalisation of the motion all likely played some part in the reduction of fidelity.
Of course, some of these processes of sufficient amplitude will destroy higher order coherence.
This is where the resilience of the certifier to always fail safe is most important; even in the presence of experimental imperfections, the observed values of the certifier still guarantee that genuine 3-coherence was created in the three- and four-element superposition tests.

For the four-element superposition, which failed to reach a value of $R_3$ sufficient to classify it as 4-coherent, we can extend the analysis a little further.
Using the coherent-state parametrisations and maximum-likelihood techniques introduced in \cref{sec:coherence-numeric-thresholds,sec:coherence-experiment-state-creation} we can optimise to find the 3-coherent state that is most likely to have produced the observed data.
We do not need to limit ourselves to the final interference pattern in this case; we can also use the blue-sideband-scan data generated during the initial tests of the superposition creation.
The results of this best state are plotted as the blue dashed line in \cref{fig:coherence-phase-4}.
This does not produce a particularly convincing fit; it is similar towards the centre of the pattern, but has significantly reduced visibility overall.
It is perhaps likely, then, that the state created was in fact 4-coherent, but the $R_3$ certifier returned only an inconclusive result.
This is not a failure of the certifier, but more a further example of its fail-safe nature; it will either describe a state as \emph{certainly $k$-coherent} or not offer an opinion.
If one is prepared to relax the burden of proof from \emph{beyond reasonable doubt} to a mere \emph{balance of probability}, the maximum-likelihood estimation offers additional information, without further experimental cost.



\section{Conclusion}

Multilevel coherence is a resource, and despite its complex definitions in terms of density operators, its presence can be certified by one-dimensional interference pattern experiments.
This remains true when the coherence basis itself is not accessible to direct measurement, such as for the motional states of a single trapped ion.
Even in cases where the certifier test is inconclusive, one can supplement the analysis with maximum-likelihood-estimation techniques to indicate whether it is probable that the observed interference pattern was caused by a state with the particular rank of coherence.
The certifier is robust against false-positives, and is massively more practical to implement than complete state tomography, especially in cases with a poor measurement operation.

We have shown that previous interference-pattern methods for coherence certification can be extended to the case of these non-ideal measurement operators, analytically for low ranks of coherence and numerically beyond that.
We have performed a detailed analysis of the statistical properties of any measurement of this certifier in an experimental setting, and shown how the bias in the standard estimator of it can be reduced.
A scheme for optimal measurement mappings was demonstrated that converted low-visibility patterns into ones that could be used to certify high-rank coherence, and these were demonstrated in a real quantum system in collaboration with the experimental group at Imperial.
Two states that unambiguously exhibited properties of 3-coherence were created and verified in the motional state of a single trapped ion.
We also provided some additional analysis that indicate higher coherence, when certification is inconclusive.

The intricacies of high-order coherence remain little understood, but this chapter has introduced a new method of probing them.
The methods described here are generally applicable far beyond trapped ions.
All that is required is a simple measurement and a phase evolution that is very naturally available to any system containing a quantum harmonic oscillator component, such as optomechanical oscillators.
This opens up the physical systems that can be used to investigate higher order coherence.
Going further, the construction of quantum computers requires incredibly detailed control over large-scale superpositions.
A method for generating the necessary phase advancement on an arbitrary number of qubits using only local operations was sketched out, allowing these interference-pattern methods to be used in these situations.

Of course, the other major requirement in quantum computing is the creation of entanglement that is robust to variations in the environment.
We now add a second ion to our theoretical trap, and move to discussing how we can generate a Bell state between them that will not suffer as heavily from environment noise.
